\documentclass{article}
\usepackage[utf8]{inputenc}
\usepackage[margin=1.0in]{geometry}
\usepackage{amsmath}
\usepackage{amssymb}
\usepackage{fancyhdr}
\usepackage{physics}
\usepackage{wrapfig}
\usepackage{hyperref}
\usepackage{multirow}
\usepackage{amsthm}
\usepackage{pgfplots}

\pgfplotsset{compat=1.16}



\renewcommand{\thesubsection}{\thesection\Alph{subsection}}
\renewcommand\qedsymbol{\square}



\title{Particle Physics PS1}
\author{Joe Crowley}
\date{October 2020}

\pagestyle{fancy}
\renewcommand{\headrulewidth}{0pt}
\renewcommand{\footrulewidth}{1pt}

\fancyhf{}
\rhead{
Joe Crowley \\
Physics 225 \\
Problem Set 2\\
}
\rfoot{Page \thepage}

\begin{document}  


\section{Questions on natural units}

\subsection{}
\textit{The cross section for the process $e^{+} e^{-} \rightarrow \mu^{+} \mu^{-}$ (via a virtual photon) is given by $\sigma=\frac{4 \pi}{3} \frac{\alpha^{2}}{E_{\mathrm{cm}}^{2}}$ in n.u., where $E_{\mathrm{cm}}$ is the total energy in the center-of-momentum frame. (This formula applies for $\left.E_{\mathrm{CM}}>>m_{\mu} .\right)$ What are the required dimensions of a cross section? Use this fact to restore any needed powers of $\hbar$ and $c$ to the cross section formula. Next, compute the numerical value of this cross section at $E_{\mathrm{CM}}=1 \mathrm{GeV}$ in the units of barns, where 1 barn $=1 \mathrm{b}=10^{-24} \mathrm{cm}^{2}$}


\subsection{}
\textit{When a low energy photon scatters from an atomic electron, there is an energy regime in which $E_{B}<<E(\gamma)<<m_{e} .$ (Here $E_{B}$ is the atomic binding energy.) This is called Thomson scattering, and the cross section is given by
$$
\sigma\left(\gamma e^{-} \rightarrow \gamma e^{-}\right)=\frac{8 \pi}{3}\left(\frac{\alpha}{m_{e}}\right)^{2}
$$
Restore any needed powers of $\hbar$ and $c,$ and then evaluate the scattering cross section in $\mathrm{cm}^{2} .$ Hint: $\sigma=6.6 \times 10^{-25} \mathrm{cm}^{2}$}


\subsection{}
\textit{Bohr radius, Compton wavelength, and "classical" electron radius. Recall that, in n.u., the Bohr radius of the H-atom is $a_{0}=1 /\left(m_{e} \alpha\right)$ and the Compton wavelength of the electron is $\lambda_{C}=1 / m_{e} .$ From the discussion of Thomson scattering, you can see that we can define a length scale $r_{0}=$ $\alpha / m_{e}$ in which $\alpha$ is in the numerator. This quantity is called the classical electron radius. Make a table listing the values of these three quantities. We will later see that the Compton wavelength is extremely important, because it sets a length scale below which quantum field theory must be used rather than a single particle wave equation (for the given type of particle}

\newpage


\section{Simple relativity: Lorentz invariant quantities}
\textit{Consider two arbitrary four vectors, $a=\left(a^{0}, \vec{a}\right)$ and $b=\left(b^{0}, \vec{b}\right)$.}

\subsection{}
\textit{Show explicitly that the dot product
$$
a \cdot b=a^{0} b^{0}-\vec{a} \cdot \vec{b}
$$
is invariant under a Lorentz transformation along the $z$ axis. In other words show that $a^{\prime} \cdot b^{\prime}=a \cdot b,$ where $a^{\prime}$ and $b^{\prime}$ are the Lorentz-transformed four vectors.}

\subsection{}
\textit{Suppose that two four-vectors are given by $a=(10.5,4.3,-5.6,7.9)$ (in some units) and $b=(-22.6,-3.5,6.6,50.4)$ (in some other units). Suppose that in another reference frame, $a^{\prime}=(7.777,4.3,-5.6,-3.556) .$ What are the parameters $\beta$ and $\beta \gamma$ for the Lorentz transformation? Compute $b^{\prime}$ and verify that $a \cdot a=a^{\prime} \cdot a^{\prime}$ and $a \cdot b=a^{\prime} \cdot b^{\prime}$}

\subsection{}
\textit{Suppose a particle has four-momentum $p=(E, \vec{p})$ in some reference frame. What is the Lorentz transformation to the CM frame, where $\vec{p}^{\prime}=0 ?$ You may choose $\vec{p}$ to point in a particular direction for convenience. In the CM frame, what is the interpretation of $E ?$ Given this, what is the interpretation of $E^{2}-\vec{p}^{2}$ in an arbitrary reference frame?}


\newpage
\section{}
\subsection{}
\textit{}

\newpage
\section{}
\subsection{}
\textit{}

\newpage
\section{}
\subsection{}
\textit{}

\newpage
\section{}
\subsection{}
\textit{}

\newpage
\section{}
\subsection{}
\textit{}

\newpage
\section{}
\subsection{}
\textit{}

\newpage
\section{}
\subsection{}
\textit{}

\newpage
\section{}
\subsection{}
\textit{}

\end{document}
